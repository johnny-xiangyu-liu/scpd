\item \subquestionpoints{5} \textbf{Coding question: the double descent phenomenon for unregularized models}

In this sub-question, you will empirically observe the double descent phenomenon.You are given $13$ training datasets of sample sizes $n  = 200, 250, \dots, 750$, and $800$, and a test dataset, located at
\begin{itemize}
	\item \texttt{src/doubledescent/train200.csv}, \texttt{train250.csv}, etc.
	\item \texttt{src/doubledescent/test.csv}
\end{itemize} 

For each training dataset $(X, \vec{y})$, compute the corresponding $\hat{\beta}_0$, and evaluate the mean squared error (MSE) of $\hat{\beta}_0$ on the test dataset. The MSE for your estimators $\hat{\beta}$ on a test dataset $(X_v, \vec{y}_v)$ of size $m$ is defined as: $$\text{MSE}(\hat{\beta}) = \frac{1}{2m} \|X_v \hat{\beta}-\vec{y}_v\|^2_2.$$


Complete the \texttt{regression} method of \texttt{src/doubledescent/doubledescent.py} which takes in a training file and a test file, and computes $\hat{\beta}_0$. You can use \texttt{numpy.linalg.pinv} to compute the pseudo-inverse.

In your writeup, include a line plot of the test losses. The x-axis is the size of the training dataset (from 200 to 800); the y-axis is the MSE on the test dataset. You should observe that the test error increases and then decreases as we increase the sample size.  

\textbf{Note:} When $n\approx d$, the test MSE could be very large. For better visualization, it is okay if the test MSE goes out of scope in the plot for some points.